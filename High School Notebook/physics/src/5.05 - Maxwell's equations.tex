% LTeX: language=it

\section{Equazioni di Maxwell}

Maxwell nel 1865 capì che il campo elettrico e il campo magnetico si influenzavano a vicenda.
Propose dunque delle equazioni che lo dimostrassero, e per questo lavorò sulle equazioni del flusso e della circuitazione di $\vec{E}$ e $\vec{B}$.

\subsection{Circuitazione di $\vec{E}$ lungo $\mathcal{L}$}

Partendo da $\oint_{\mathcal{L}} \vec{E} = 0$, Maxwell dimostrò che:

\begin{equation*}
    \oint_{\mathcal{L}} \vec{E} = \sum_{i = 1}^{n} \vec{E_i} \cdot \Delta {\vec{l_i}} = \sum_{i = 0}^{n} \frac{\vec{F_i}}{q} \cdot \Delta {\vec{l_i}} = \frac{W}{q} = \text{fem}(\vec{B})
\end{equation*}

Da cui:

\begin{equation*}
    \oint_{\mathcal{L}} \vec{E} = - \frac{\Delta \Phi_S(\vec{B})}{\Delta t}
\end{equation*}

Si comprese dunque che un campo elettrico poteva essere generato da cariche elettriche, come elettroni o protoni, o da campi magnetici variabili nel tempo.

\subsection{Circuitazione di $\vec{B}$ lungo $\mathcal{L}$}

Partendo da $\oint_{\mathcal{L}} \vec{B} = \mu_0 \cdot I_{concatenata}$, Maxwell dimostrò l'esistenza di una corrente di spostamento $I_s = \varepsilon_0 \cdot \frac{\Delta \Phi_S(\vec{E})}{\Delta t}$, che si doveva sommare alla corrente $I_c$.
Quindi:

\begin{equation*}
    \oint_{\mathcal{L}} \vec{B} = \mu_0 \cdot (I_c + I_s) = \mu_0 \cdot \left( I_c + \varepsilon_0 \cdot \frac{\Delta \Phi_S(\vec{E})}{\Delta t} \right)
\end{equation*}

Si comprese dunque che un campo magnetico poteva essere generato da correnti elettriche come nel caso di spire percorse da correnti, o da campi elettrici variabili nel tempo.

\subsection{Definizione di campo elettromagnetico}

Se definì così il campo elettromagnetico, dove $\vec{E}$ e $\vec{B}$ si influenzavano a vicenda senza bisogno di alcun mezzo materiale.

\begin{align*}
    \Phi_\Omega(\vec{E})        & = \frac{Q_{tot}}{\varepsilon}                                                                  \\
    \Phi_\Omega(\vec{B})        & = 0                                                                                            \\
    \oint_{\mathcal{L}} \vec{E} & = - \frac{\Delta \Phi_S(\vec{B})}{\Delta t}                                                    \\
    \oint_{\mathcal{L}} \vec{B} & = \mu_0 \cdot \left( I_c + \varepsilon_0 \cdot \frac{\Delta \Phi_S(\vec{E})}{\Delta t} \right)
\end{align*}

La risoluzione delle equazioni di Maxwell porta alla definizione di onde elettromagnetiche come la luce.
