% LTeX: language=it

\section{Relatività ristretta}

\subsection{Velocità della luce e sistemi di riferimento}

Il valore $c = 3 \cdot 10^8 \si{\meter\per\second}$ è un valore massimo e assoluto che non dipende dal sistema di riferimento (S.R.) considerato.
Si ricorda che un S.R.I. (inerziale) è un sistema che di muove di moto rettilineo uniforme e quindi senza accelerazioni.
Essendo $c$ un invariante, allora o la meccanica di Newton/Galileo o l'elettromagnetismo sono sbagliate.
La relatività corregge le formule della meccanica classica.

\subsubsection{Esempio di uno shuttle in movimento che spara due raggi laser}

% TODO: disegno dello shuttle con interpretazione classica della velocità (v+c, v-c) e relativistica (c)

\subsection{L'esperimento di Michelson-Morley (1887)}

Aveva lo scopo di dimostrare l'effetto del vento d'etere sulla velocità della luce $c$.

% TODO: disegno del vento d'etere con sole e Terra orbitante

La Terra compiendo un moto di rivoluzione "naviga" in questa sostanza, in $T_1$ contro vento e in $T_2$ a favore.
Quindi secondo Galileo $c_1 < c$ perché influenzato dal vento, e $c_2 > c$ perché velocizzato dall'etere.
Per osservare la velocità, i fisici costruirono un'interferometro costituito da diversi specchi e una lastra semi-argentata che divide il raggio $1$.

% TODO: disegno dell'interferometro

Il raggio $R_1$  si rifrange su $S_{p-s}$ e si divide in $R_2$ e $R_3$.
Uno dei due viaggia controvento o a favore e quindi $\Delta t_2 = \frac{l}{v+c} + \frac{l}{v-c}$ = $\frac{2l}{c} \cdot \frac{1}{1-\frac{v^2}{c^2}}$ e $\Delta t_3 = \frac{2l}{\sqrt{c^2-v^2}} = \frac{2l}{c} \cdot \frac{1}{\sqrt{1-\frac{v^2}{c^2}}}$.

Si ottiene così una ipotetica fase tra i due raggi di:

\begin{equation*}
    \Delta \phi_{ipotesi} = \Delta t_2 - \Delta t_3 = \frac{2l}{c} \cdot \left( \frac{1}{1-\frac{v^2}{c^2}} - \frac{1}{\sqrt{1-\frac{v^2}{c^2}}} \right)
\end{equation*}

Essendo $R_4 = R_2 + R_3$, se c'è una certa fase allora si creeranno dei fronti so'onda su S.
Da questi si ricava il $v$ del vento d'etere.

Risultato negativo: $c$ non varia quindi l'etere on influenzava la luce.

\subsection{Gli assiomi della teoria della relatività ristretta}

\begin{itemize}
    \item Le leggi e i principi della fisica hanno la stessa forma in tutti i sistemi di riferimento inerziali (S.R.I.).
    \item La velocità della luce $c$ nel vuoto è la stessa in tutti i S.R.I., indipendentemente dal moto del sistema o della sorgente da cui la luce è emessa.
\end{itemize}

\subsection{La simultaneità}

Definizione operativa: due eventi $E_1 / E_2$ sono simultanei quando, preso $P_3 = \frac{P_1 + P_2}{2}$, il segnale luminoso emesso da $E_1$ impiega lo stesso $\Delta t_1$ per raggiungere $P_3$ del segnale luminoso emesso da $E_2$, $\Delta t_2 \rightarrow $

% TODO: disegno della simultaneità (omino e segnali acustici)

La simultaneità diventa però relativa se consideriamo più di un solo S.R.I.

\subsubsection{Esempio del treno e dei lampi di luce}

Presi $E_1$ e $E_2$  simultanei per il sistema di riferimento $O_{terra}$, essi non sono più simultanei per il sistema di riferimento $O_{movimento}$.

% TODO: disegno del treno con lampi di luce

\subsection{La dilatazione dei tempi}

Due orologi sono sincronizzati se al tempo $t_0$, l'orologio $1$ lancia un segnale luminoso all'orologio $2$, e questo, quando il raggio lo raggiunge segna $t = t_0 + \frac{d}{c}$

% TODO: disegno degli orologi e della distanza tra i due

Se però abbiamo due S.R.I. in moto relativo, allora il tempo $t$ non scorre più alla stessa velocità tra i due sistemi, ma che nel sistema in movimento il tempo scorre più lentamente rispetto a quello fermo.

\subsubsection{Esempio dei laser in movimento}

% TODO: disegno dei laser in movimento

Dai disegni sopra, si nota come il cammino percorso dal laser cambi in base alla posizione dell'osservatore.
In particolare:

\begin{itemize}
    \item Osservatore a bordo del carrello laser (in movimento): $t = \frac{2d}{c}$
    \item Osservatore fermo: $t' = \frac{2d}{c} \cdot \frac{1}{\sqrt{1-\frac{v^2}{c^2}}}$
\end{itemize}

Si ha quindi che $\Delta t' > \Delta t$ e quindi che il tempo in un S.R.I. in movimento rispetto a un altro scorre più lentamente.

Si definiscono allora:

\begin{align*}
    \Delta t  & = \text{tempo proprio}                                                        \\
    \Delta t' & = \text{tempo improprio}                                                      \\
    \gamma    & = \frac{1}{\sqrt{1-\frac{v^2}{c^2}}} \quad \text{coefficiente di dilatazione} \\
    \beta     & = \frac{v}{c}
\end{align*}

\subsection{La contrazione delle lunghezze}

Come conseguenza della dilatazione dei tempi, vi è anche la contrazione delle lunghezze.
Essendo $\Delta x = v \cdot \Delta t'$ e $\Delta x' = v \cdot \Delta t$, si ha che:

\begin{equation*}
    \Delta x' = \frac{\Delta x}{\gamma} = \Delta x \cdot \sqrt{1-\frac{v^2}{c^2}}
\end{equation*}

Dove $\Delta x$ è la lunghezza propria del segmento e $\Delta x'$ è la lunghezza impropria, ovvero la lunghezza del segmento visto dall'esterno e da fermo.

Abbiamo quindi che $\Delta t' = \gamma \cdot \Delta t$ e $\Delta x' = \frac{\Delta x}{\gamma}$, con $\gamma$ avente la seguente forma:

\begin{figure}[H]
    \centering
    \begin{tikzpicture}

        \begin{axis}[
                axis lines = left,
                xlabel = $v$,
                ylabel = {$\gamma$},
                xmin = 0,
                xmax = 1,
                ymin = 0,
                ymax = 10,
            ]

            % \node at (axis cs:0.5,-0.5) {$c$};

            % Below the red parabola is defined
            \addplot [
                domain=0:1,
                samples=100,
                color=red,
            ]
            {1/sqrt(1-x^2)};
            \addlegendentry{$\gamma = \frac{1}{\sqrt{1-\frac{v^2}{c^2}}}$}

        \end{axis}

    \end{tikzpicture}
\end{figure}

\subsubsection{Esempio del caso del muone}

Il muone è una particella che si forma nella parte alta dell'atmosfera e ha una vita media di $\Delta t = 2.2 \cdot 10^{-6} \si{\second}$.
Data la sua elevata velocità $v = 0.998c$, per valutarne la possibile distanza $\Delta x$ (vista dalla Terra) che può percorrere prima di decadere, si deve utilizzare la correzione relativistica.
Si ha cosi che:

\begin{figure}[H]
    \begin{minipage}[t]{0.45\textwidth}
        \centering
        Meccanica classica
        \begin{equation*}
            \Delta x = v \cdot \Delta t = 0.998c \cdot 2.2 \cdot 10^{-6} \approx 660 \si{\meter}
        \end{equation*}
    \end{minipage}
    \hfill
    \begin{minipage}[t]{0.45\textwidth}
        \centering
        Relatività ristretta
        \begin{align*}
            \gamma    & = \frac{1}{\sqrt{1-\frac{v^2}{c^2}}} = \frac{1}{\sqrt{1-\frac{(0.998c)^2}{c^2}}} = 15                    \\
            \Delta t' & = \gamma \cdot \Delta t = 15 \cdot 2.2 \cdot 10^{-6} \approx 35 \si{\micro\second}                       \\
            \Delta x  & = \gamma \cdot \Delta x' = v \cdot \Delta t' = 0.998c \cdot 35 \cdot 10^{-6} \approx 10 \si{\kilo\meter}
        \end{align*}
    \end{minipage}
\end{figure}

Da notare che con $\Delta x$ abbiamo calcolato la distanza che il muone può percorrere prima di decadere vista da un osservatore fermo (sulla Terra), mentre con $\Delta x'$ abbiamo calcolato la distanza che il muone può percorrere prima di decadere vista da un osservatore in moto insieme al muone.

In altre parole, al muone sembrerà di aver percorso una distanza decisamente minore rispetto a quella calcolata dall'osservatore fermo (contrazione delle lunghezze).
In maniera duale, al muone sembrerà di essere in vita per un tempo decisamente maggiore rispetto a quello calcolato dall'osservatore fermo (dilatazione dei tempi).

In definitiva, si ha dunque che se per il muone la distanza che percorre è solo di $660 \si{\meter}$, dalla terra il muone percorre una distanza di circa $10 \si{\kilo\meter}$.

\subsection{L'invarianza delle lunghezze perpendicolari al moto relativo}

Si sottolinea ora che la contrazione delle lunghezze avviene solo per le lunghezze parallele al moto relativo, ovvero lungo la direzione del moto.
Per le lunghezze perpendicolari al moto, invece, non vi è alcuna contrazione.

\subsubsection{Esempio del treno e del tunnel}

Un esempio classico che prova logicamente l'invarianza delle lunghezze perpendicolari al moto relativo è quello del treno che attraversa un tunnel.
É infatti un assurdo pensare che il passaggio o meno del treno all'ingresso della galleria dipenda dalla velocità del treno stesso.

\subsection{Le trasformazioni di Lorentz}

Sono delle equazioni per ottenere le coordinate spazio-temporali di un punto materiale in un S.R.I. ($S'$), partendo dalle coordinate di un altro S.R.I. ($S$).
Le trasformazioni di Galileo vengono qui sostituite da quelle di Lorentz:

\begin{figure}[H]
    \begin{minipage}[t]{0.45\textwidth}
        \centering
        Galileo
        \begin{align*}
            x' & = x - vt \\
            y' & = y      \\
            z' & = z      \\
            t' & = t
        \end{align*}
        % TODO: disegno delle trasformazioni di Galieleo
    \end{minipage}
    \hfill
    \begin{minipage}[t]{0.45\textwidth}
        \centering
        Lorentz (dirette e inverse)
        \begin{align*}
            x' & = \gamma \cdot (x - vt)                             \\
            y' & = y                                                 \\
            z' & = z                                                 \\
            t' & = \gamma \cdot \left( t - \frac{\beta}{c} x \right)
        \end{align*}
        \begin{align*}
            x & = \gamma \cdot (x' + vt')                             \\
            t & = \gamma \cdot \left( t' + \frac{\beta}{c} x' \right)
        \end{align*}
    \end{minipage}
    % TODO: disegno delle trasformazioni di Lorentz
\end{figure}

Dove ogni grandezza avente l'apice è relativa al S.R.I. $S'$, mentre quelle senza apice sono relative al S.R.I. $S$.

\subsection{L'intervallo invariante}

Se consideriamo uno spostamento nello spazio, esso può essere rappresentato in infiniti piani cartesiani, con stessa origine $O$.

Per entrambe le rappresentazioni avremo $\Delta s^2 = \Delta x^2 + \Delta y^2 = \Delta X^2 + \Delta Y^2$, e quindi si può vedere che nonostante il S.R. considerato, l'intervallo $\Delta s^2$ è una misura invariante.

In fisica, consideriamo la quaterna ordinata $(x, y, z, t)$ con il nome di "Evento".
Avendo quindi due eventi $E_1$ e $E_2$ in un dato S.R., possiamo calcolare l'intervallo invariante come:

\begin{equation*}
    \Delta \sigma^2 = (c\Delta t)^2 - \Delta x^2 - \Delta y^2 - \Delta z^2 = (c\Delta t)^2 - \Delta s^2
\end{equation*}

Dove $\Delta s^2$ è la distanza spaziale tra i due eventi e $\Delta \sigma^2$ è l'intervallo invariante tra i due eventi.

In base al segno di $\Delta \sigma^2$, possiamo classificare gli eventi in:

\begin{itemize}
    \item $\Delta \sigma^2 > 0$: Eventi casualmente connessi. Se $E_1$ lancia un segnale, esso riesce a raggiungere $E_2$. Si può quindi dire che esiste un S.R. dove $\Delta s = 0$ e $\Delta t = \frac{\sqrt{\Delta s^2}}{c}$.
    \item $\Delta \sigma^2 < 0$: Eventi casualmente non connessi. Un segnale lanciato da $E_1$ non può raggiungere $E_2$. Si può quindi dire che esiste un S.R. dove $\Delta s = \sqrt{-\Delta \sigma^2}$.
    \item $\Delta \sigma^2 = 0$: Solo un segnale luminoso a velocità $c$ può collegare i due eventi, ovvero raggiungere $E_2$ partendo da $E_1$.
\end{itemize}

\subsection{Lo spazio-tempo di Minkowski}

Ad ogni Evento, oltre che definirlo tramite le coordinate spaziali (x, y, z), va definito da un quadrivettore spazio-temporale $(x, y, z, t)$, dove quindi alla definizione 3D se ne aggiunge una, diventando una definizione spazio-tempo in 4D.
Gli eventi sono definiti da $\Delta \sigma$.

Il diagramma di Minkowski è la rappresentazione grafica di eventi, e gli assi sono definiti come $ct(x)$ o $ct(y)$ o $ct(z)$, mentre l'evento $E$ è rappresentato come punto $E(x_0, ct_0)$.

% TODO: disegno del diagramma di Minkowski

Se per esempio consideriamo un raggio di luce che parte in $x_0 = -3m$ al tempo $t_0 = 1s$ e viaggia a $v = c$, allora rispetto a un uomo fermo in $x_0 = 0m$, avremo:

\begin{figure}[H]
    \centering
    \begin{tikzpicture}

        \begin{axis}[
                axis lines = left,
                xlabel = $x$,
                ylabel = {$ct$},
                xmin = -5,
                xmax = 5,
                ymin = 0,
                ymax = 5,
            ]

            \coordinate (E0) at (-3,1);
            \coordinate (E) at (0,4);

            \draw[dashed] (E0) -> (E)+(0.1,0.4);
            \draw (E0) node[below, left] {$E_0$};

        \end{axis}
    \end{tikzpicture}
    \caption{Diagramma di Minkowski applicato al caso 1D}
\end{figure}

% TODO: disegno dell'omino e del raggio di luce

Grazie al diagramma di Minkowski è quindi possibile calcolare dove si troverà il fascio dopo una certo tempo $\Delta t$ o un certo spazio $\Delta x$.

\subsection{La composizione relativistica delle velocità}

Così come per le coordinate, anche per le velocità Galileo non è più corretto.
Per calcolare $u'$, ovvero la velocità di $E$ ma rispetto a $S'$, si ha che:

\begin{figure}[H]
    \begin{minipage}[t]{0.45\textwidth}
        \centering
        Galileo
        \begin{equation*}
            u' = u - v
        \end{equation*}
    \end{minipage}
    \hfill
    \begin{minipage}[t]{0.45\textwidth}
        \centering
        Lorentz (dirette e inverse)
        \begin{equation*}
            u' = \frac{u - v}{1 - \frac{uv}{c^2}}
        \end{equation*}
        \begin{equation*}
            u = \frac{u' + v}{1 + \frac{uv}{c^2}}
        \end{equation*}
    \end{minipage}
\end{figure}

\subsubsection{Esempio dell'astronave e della pallottola}

Per comprendere meglio il concetto di composizione relativistica delle velocità, consideriamo il seguente esempio.
Supponiamo di avere un'astronave che viaggia a $u$ e che spara una pallottola a velocità $v$, entrambe misurate da un osservatore fermo sulla Terra.
Per calcolare la velocità della pallottola rispetto all'astronave, avremo che $v' = \frac{v-u}{1-\frac{uv}{c^2}}$ e non più $v' = v-u$ come in meccanica classica.

% TODO: disegno dell'astronave e della pallottola

\subsection{L'equivalenza tra massa ed energia}

Con relatività si associa energia alla massa, infatti $\Delta E = \Delta m \cdot c^2$.
La massa e l'energia prese singolarmente non sono più invarianti, ma ora vale il principio della conservazione massa-energia.

La massa a riposo $m_0$, è la massa caratteristica del corpo quando $v = 0$, mentre la massa relativistica $m$ è la massa del corpo in movimento.

Si ha quindi che $E_0 = m_0 \cdot c^2$ e $E = m \cdot c^2$, con $m = \gamma \cdot m_0$.

\subsection{La dinamica relativistica}

Consideriamo ora corpi in movimento a velocità $v$, possiamo affermare che l'energia totale è $E = \gamma m_0 c^2$, che $K = (\gamma - 1) m_0 c^2$, che $p_r = \gamma m_0 v$  e dunque si capisce che in un sistema isolato, a conservarsi è il quadrivettore energia-quantità di moto, in quanto:

\begin{equation*}
    m_0^2 c^2 = \frac{E^2}{c^2} - p_x^2 - p_y^2 - p_z^2 = \text{invariante}
\end{equation*}

Dal grafico di $E(v)$, risulta chiaro che è impossibile portare un corpo con $m_0 > 0$ alla velocità della luce $c$, in quanto servirebbe energia infinita.

L'unica cosa che viaggia alla velocità della luce $c$ è la luce stessa, ovvero i fotoni, che infatti non hanno massa $\rightarrow E = \gamma m_0 c^2 = \gamma \cdot 0 \cdot c^2 = 0$.

\begin{figure}[H]
    \begin{minipage}[t]{0.45\textwidth}
        \centering
        \begin{tikzpicture}[scale=0.8]

            \begin{axis}[
                    axis lines = left,
                    xlabel = $v$,
                    ylabel = $E$,
                    xmin = 0,
                    xmax = 1,
                    ymin = 0,
                    ytick=\empty,
                    % extra y ticks={\infty},
                    extra y tick labels=$+\infty$,
                ]

                % \node at (axis cs:0.5,-0.5) {$c$};

                % Below the red parabola is defined
                \addplot [
                    domain=0:1,
                    samples=100,
                    color=red,
                ]
                {1/sqrt(1-x^2) * 1 * 3^2};
                \addlegendentry{$E(v, m_0 > 0)$}

            \end{axis}

        \end{tikzpicture}
        \caption{Grafico di $E$ in funzione di $v$ per $m_0 > 0$}
    \end{minipage}
    \hfill
    \begin{minipage}[t]{0.45\textwidth}
        \centering
        \begin{tikzpicture}[scale=0.8]

            \begin{axis}[
                    axis lines = left,
                    xlabel = $v$,
                    ylabel = $E$,
                    xmin = 0,
                    xmax = 1,
                    ymin = 0,
                    ymax = 1,
                ]

                % \node at (axis cs:0.5,-0.5) {$c$};

                % Below the red parabola is defined
                \addplot [
                    domain=0:1,
                    samples=100,
                    color=red,
                ]
                {0};
                \addlegendentry{$E(v, m_0 = 0)$}

            \end{axis}

        \end{tikzpicture}
        \caption{Grafico di $E$ in funzione di $v$ per $m_0 = 0$}
    \end{minipage}
\end{figure}