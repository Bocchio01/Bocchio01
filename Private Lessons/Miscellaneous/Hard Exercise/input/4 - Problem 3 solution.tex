% LTeX: language=it

\subsection{Problema 3}
Osserviamo graficamente il problema dato e notiamo si tratta di un limite elementare non essendoci punti di discontinuità:
\begin{center}
    \begin{tikzpicture}
        \begin{axis}[
                axis lines=middle,
                xmin=0,xmax=10,
                ymin=-0.1,ymax=0.2,
                legend style={anchor=north west},
            ]
            \addplot[blue, samples=200, domain=0.01:10] {ln(x^(4*sqrt(3)/73))};
            \addlegendentry{$\ln(x^{\frac{4\sqrt{3}}{73}})$}
            \addplot[red, samples=200, domain=0.01:10] {4*sqrt(3)/(73*x)};
            \addlegendentry{$\frac{d}{dx}\ln(x^{\frac{4\sqrt{3}}{73}}) = \frac{4\sqrt{3}}{73x}$}
            \addplot[only marks, mark=*] coordinates {(8.66, 4/365)};
        \end{axis}
    \end{tikzpicture}
\end{center}

\noindent Si procede ora al calcolo del limite:
\begin{equation}
    \lim_{x \rightarrow 5\sqrt{3}} \frac{d}{dx} \ln(x^{\frac{4\sqrt{3}}{73}})
\end{equation}
Si procede ora al calcolo della derivata:
\begin{equation}
    \frac{d}{dx} \ln(x^{\frac{4\sqrt{3}}{73}}) = \frac{4\sqrt{3}}{73} \frac{d}{dx} \ln(x) = \frac{4\sqrt{3}}{73} \frac{1}{x}
\end{equation}
Si procede ora al calcolo del limite:
\begin{equation}
    \lim_{x \rightarrow 5\sqrt{3}} \frac{4\sqrt{3}}{73} \frac{1}{x} = \frac{4\sqrt{3}}{73} \frac{1}{5\sqrt{3}} = \frac{4}{365}
\end{equation}