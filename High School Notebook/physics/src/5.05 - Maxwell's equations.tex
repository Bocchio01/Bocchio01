% LTeX: language=it

\section{Equazioni di Maxwell}

Nel 1865, Maxwell comprese che il campo elettrico e il campo magnetico si influenzavano reciprocamente.
Egli propose delle equazioni basate sul flusso e sulla circuitazione di $\vec{E}$ e $\vec{B}$ per dimostrarlo.

\subsection{Circuitazione di $\vec{E}$ lungo $\mathcal{L}$}

Partendo da $\oint_{\mathcal{L}} \vec{E} = 0$, Maxwell dimostrò che tale circuitazione è legata alla forza elettromotrice $\text{fem}(\vec{B})$ attraverso:

\begin{equation*}
    \oint_{\mathcal{L}} \vec{E} = - \frac{\Delta \Phi_S(\vec{B})}{\Delta t}
\end{equation*}

Questo indica che un campo elettrico può essere generato da cariche elettriche o da campi magnetici variabili nel tempo.

\subsection{Circuitazione di $\vec{B}$ lungo $\mathcal{L}$}

Partendo da $\oint_{\mathcal{L}} \vec{B} = \mu_0 \cdot I_{\text{concatenata}}$, Maxwell introdusse la corrente di spostamento $I_s = \varepsilon_0 \cdot \frac{\Delta \Phi_S(\vec{E})}{\Delta t}$.
Quindi, la circuitazione diventa:

\begin{equation*}
    \oint_{\mathcal{L}} \vec{B} = \mu_0 \cdot \left( I_c + I_s \right) = \mu_0 \cdot \left( I_c + \varepsilon_0 \cdot \frac{\Delta \Phi_S(\vec{E})}{\Delta t} \right)
\end{equation*}

Questo dimostra che un campo magnetico può essere generato da correnti elettriche o da campi elettrici variabili nel tempo.

\subsection{Definizione di campo elettromagnetico}

La combinazione delle equazioni di Maxwell porta alla definizione del campo elettromagnetico, in cui $\vec{E}$ e $\vec{B}$ si influenzano reciprocamente senza la necessità di un mezzo materiale.

\begin{align*}
    \Phi_\Omega(\vec{E})        & = \frac{Q_{\text{tot}}}{\varepsilon}                                                           \\
    \Phi_\Omega(\vec{B})        & = 0                                                                                            \\
    \oint_{\mathcal{L}} \vec{E} & = - \frac{\Delta \Phi_S(\vec{B})}{\Delta t}                                                    \\
    \oint_{\mathcal{L}} \vec{B} & = \mu_0 \cdot \left( I_c + \varepsilon_0 \cdot \frac{\Delta \Phi_S(\vec{E})}{\Delta t} \right)
\end{align*}

La risoluzione di queste equazioni conduce alla definizione delle onde elettromagnetiche, come ad esempio la luce.
