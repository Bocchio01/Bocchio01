% LTeX: language=it

\subsection{Problema 2}
Come prima, è possibile visualizzare i due vettori e il vettore prodotto in un piano cartesiano tridimensionale.
Si sottolinea come la figura seguente non rappresenti la soluzione del problema, ma sia solo un aiuto visivo per la comprensione del problema.
\begin{center}
    \begin{tikzpicture}[x=1cm, y=1cm, z=-0.6cm]
        % Axes
        \draw [->] (-3,0,0) -- (4,0,0) node [right] {$x$};
        \draw [->] (0,0,0) -- (0,4,0) node [left] {$y$};
        \draw [->] (0,0,0) -- (0,0,4) node [left] {$z$};
        % Vectors
        \draw [->, thick] (0,0,0) -- (2,2,0);
        \draw [->, thick] (0,0,0) -- (2,0,1);
        \draw [->, thick, red] (0,0,0) -- (-2,2,4);
        % Ticks
        \foreach \i in {1, 2, 3}
            {
                \draw (-0.1,\i,0) -- ++ (0.2,0,0);
                \draw (\i,-0.1,0) -- ++ (0,0.2,0);
                \draw (-0.1,0,\i) -- ++ (0.2,0,0);
            }
        \draw (-1,-0.1,0) -- ++ (0,0.2,0);
        \draw (-2,-0.1,0) -- ++ (0,0.2,0);
        % Dashed lines
        \draw [loosely dashed]
        (0,2,0) -- (2,2,0) -- (2,0,0)
        (0,0,1) -- (2,0,1) -- (2,0,0)
        (0,0,4) -- (-2,0,4) -- (-2,2,4)
        (-2,0,0) -- (-2,0,4)
        ;
        % Labels
        \node [right] at (2,2,0) {$\begin{bmatrix}
                    2 \\2\\0
                \end{bmatrix}$};
        \node [below] at (2,0,1) {$\begin{bmatrix}
                    2 \\0\\1
                \end{bmatrix}$};
        \node [above] at (-2,2,4) {$\begin{bmatrix}
                    -2 \\2\\4
                \end{bmatrix}$};

    \end{tikzpicture}
\end{center}

\noindent Si procede ora al calcolo del prodotto vettoriale tra $\vec{a}$ e $\vec{b}$:
\begin{equation}
    \vec{a} \times \vec{b} = \begin{vmatrix}
        \vec{i} & \vec{j}     & \vec{k}               \\
        1       & \frac{1}{3} & 3+6\sqrt{\frac{2}{5}} \\
        3       & 1           & 9
    \end{vmatrix} = \vec{i} \begin{vmatrix}
        \frac{1}{3} & 3+6\sqrt{\frac{2}{5}} \\
        1           & 9
    \end{vmatrix} - \vec{j} \begin{vmatrix}
        1 & 3+6\sqrt{\frac{2}{5}} \\
        3 & 9
    \end{vmatrix} + \vec{k} \begin{vmatrix}
        1 & \frac{1}{3} \\
        3 & 1
    \end{vmatrix}
\end{equation}
Si procede ora al calcolo dei determinanti:
\begin{equation}
    \begin{split}
        \begin{vmatrix}
            \frac{1}{3} & 3+6\sqrt{\frac{2}{5}} \\
            1           & 9
        \end{vmatrix} = \frac{1}{3} * 9 - (3+6\sqrt{\frac{2}{5}}) = -6\sqrt{\frac{2}{5}} \\
        \begin{vmatrix}
            1 & 3+6\sqrt{\frac{2}{5}} \\
            3 & 9
        \end{vmatrix} = 1 * 9 - 3(3+6\sqrt{\frac{2}{5}}) = -18\sqrt{\frac{2}{5}} \\
        \begin{vmatrix}
            1 & \frac{1}{3} \\
            3 & 1
        \end{vmatrix} = 1 * 1 - \frac{1}{3} * 3 = 1 - 1 = 0
    \end{split}
\end{equation}
Si procede ora al calcolo del prodotto vettoriale:
\begin{equation}
    \vec{a} \times \vec{b} = \vec{i} (-6\sqrt{\frac{2}{5}}) - \vec{j} (-18\sqrt{\frac{2}{5}}) + \vec{k} * 0 = (-6\sqrt{\frac{2}{5}})\vec{i} + (18\sqrt{\frac{2}{5}})\vec{j}
\end{equation}
Si procede ora al calcolo del modulo del prodotto vettoriale:
\begin{equation}
    |\vec{a} \times \vec{b}| = \sqrt{(-6\sqrt{\frac{2}{5}})^2 + (18\sqrt{\frac{2}{5}})^2} = \sqrt{36\frac{2}{5} + 324\frac{2}{5}} = \sqrt{\frac{720}{5}} = \sqrt{144} = 12
\end{equation}