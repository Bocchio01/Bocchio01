% LTeX: language=it

\section{Esponenziali e logaritmi}

Si definisce funzione esponenziale ogni funzione del tipo $f(x) = a^x$, con $a \in \mathbb{R}$ e $a \geq 0$.

Se $a > 1$, la funzione è sempre crescente, mentre se $0 < a < 1$ è sempre decrescente.

% TODO: grafici funzioni esponenziali

É un equazione esponenziale, una qualsiasi equazione che contiene almeno una potenza con l'incognita all'esponente: $a^x = b$, risolvibile con il logaritmo $x = \log_a(b)$.
Per risolvere le disequazioni con gli esponenziali, si riporta tutto alla stessa base e si osserva se:
\begin{itemize}
    \item $a > 1$, allora si pongono gli esponenti uno maggiore dell'altro $\rightarrow 2^{2x} > 2^3$, Soluzione: $2x > 3$
    \item $0 < a < 1$, allora si pongono gli esponenti uno minore dell'altro $\rightarrow \frac{1}{3}^{2x} > \frac{1}{3}^5$, Soluzione: $2x < 5$
\end{itemize}

Il logaritmo è l'esponente da dare alla base per ottenere l'argomento: $\log_a(b) = x \iff a^x = b$.
Le condizioni di esistenza sono $b > 0$ e $a \neq 1 \wedge a > 0$

% TODO: grafici funzioni logaritmiche

\subsection{Proprietà dei logaritmi}

\begin{align*}
    \log_a(b \cdot c)   & = \log_a(b) + \log_a(c)       \\
    \log_a(\frac{b}{c}) & = \log_a(b) - \log_a(c)       \\
    \log_a(b^c)         & = c \cdot \log_a(b)           \\
    \log_a(b)           & = \frac{\log_c(b)}{\log_c(a)}
    \label{eq:logaritmi_proprieta}
\end{align*}

Una equazione è logaritmica se compare l'incognita nell'argomento del logaritmo: $\log_a(x)$.
Per risolvere le disequazioni, riportiamo il entrambi i membri alla stessa base, sapendo che $\log_a(b) = e^{\ln(\log_a(b))}$, oppure sfruttando la definizione di logaritmo: $a^x = b \iff x = \log_a(b)$.

\begin{gather}
    \log_3(x+2) \geq 5 \rightarrow C.E.: x+2 > 0 \\
    x+2 \geq 3^5 \rightarrow x \geq 3^5 - 2
\end{gather}

Per risolvere un equazione esponenziale, si utilizzano i logaritmi e le sue proprietà:

\begin{gather}
    7\cdot5^{2x} = 3^{x+1} \\
    \log_10(7\cdot5^{2x}) = \log_10(3^{x+1}) \\
    \log_10(7) + 2x\log_10(5) = (x+1)\log_10(3) \\
    x = \frac{\log_10(3) - \log_10(7)}{2\log_10(5) - \log_10(3)}
\end{gather}