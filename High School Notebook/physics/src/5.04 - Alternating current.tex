% LTeX: language=it

\section{Corrente Alternata}

La corrente alternata (CA) è generata da una tensione che varia continuamente, ripetendosi dopo un periodo $T$.
L'alternatore è lo strumento responsabile della produzione della CA, il cui simbolo è $AC$.

Le leggi fondamentali della CA sono espresse dalle seguenti equazioni:

\begin{align*}
    \text{fem}(t) & = f_0 \sin(\omega t), \quad f_0 = B \cdot S \cdot \omega                                               \\
    i(t)          & = i_0 \sin(\omega t), \quad i_0 = \frac{f_0}{R}                                                        \\
    \text{fem}    & = - \frac{d \Phi(\vec{B})}{d t} = BS\omega\sin(\omega t), \quad \text{(se generata da un alternatore)}
\end{align*}

\subsection{Valori Efficaci e Valori Massimi}

La potenza istantanea $P(t) = R[i(t)]^2$ conduce a una potenza media $P_{\text{media}} = \frac{1}{2} R i_0^2$.
Definiamo i valori efficaci come $i_{\text{eff}} = \frac{i_0}{\sqrt{2}}$ e $fem_{\text{eff}} = \frac{f_0}{\sqrt{2}}$.
La potenza media assorbita da un circuito $AC$ è quindi $P = i_{\text{eff}} \cdot fem_{\text{eff}}$.

Per la corrente domestica in Italia ($50\,Hz$), con $fem_{\text{eff}} = 220\,V$, otteniamo $fem_0 = 325\,V$.

\subsection{Trasformatore}

Il trasformatore è un dispositivo che modifica i valori di tensione e corrente attraverso due bobine collegate da un conduttore.
Il suo funzionamento è limitato alla corrente alternata e segue la relazione:

\begin{equation*}
    \frac{i_{2,\text{eff}}}{i_{1,\text{eff}}} = \frac{f_{1,\text{eff}}}{f_{2,\text{eff}}} = \frac{N_2}{N_1}
\end{equation*}

Dove $1$ indica l'entrata/primario e $2$ l'uscita/secondario, mentre $N$ rappresenta il numero di spire del solenoide.

% TODO: Aggiungere il circuito del trasformatore
