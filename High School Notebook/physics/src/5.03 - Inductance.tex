% LTeX: language=it

\section{Induttanza}

L'autoinduzione è il fenomeno per cui si genera una corrente indotta alla chiusura del circuito, che va a contrastare la corrente immessa.
Quando il circuito si chiude, la corrente $\vec{i}$ che inizia a circolare genera un campo magnetico $\vec{\Delta B}$ come se fosse una spira.
La legge di Lenz afferma che, avendo $\vec{\Delta B}$, si crea un $\vec{B_{indotto}}$, e di conseguenza una corrente indotta che contrasta $\vec{i}$.
Questo fenomeno è ben visibile nei circuiti $RL$.

Definiamo l'induttanza come $L = \mu_0 \mu_r \frac{N^2}{l} S$, misurabile in Henry ($H \rightarrow \frac{Wb}{A}$). Per un solenoide, all'interno si ha $\Phi(\vec{B}) = L \cdot i$, e avendo un $\Delta \Phi(\vec{B})$, possiamo determinare il valore della corrente che circola nel circuito usando le seguenti espressioni:

\begin{align*}
    \text{Chiusura} \rightarrow i(t) & = \frac{\text{fem}}{R} \left( 1 - e^{-\frac{R}{L} t} \right) \\
    \text{Apertura} \rightarrow i(t) & = \frac{\text{fem}}{R} e^{-\frac{R}{L} t}
\end{align*}

% TODO: Aggiungere un diagramma del circuito RL

\subsection{Mutua induttanza}

Quando la corrente indotta ha origine esterna (in un secondo circuito), si ha un fenomeno di mutua induttanza.

\begin{align*}
    \begin{aligned}
        \Phi_2(\vec{B_1}) & = M \cdot i_1 \\
        \Phi_1(\vec{B_2}) & = M \cdot i_2
    \end{aligned}
    \quad &
    \begin{aligned}
        \text{fem}^{1 \rightarrow 2} & = - M \frac{d i_1}{d t} \\
        \text{fem}^{2 \rightarrow 1} & = - M \frac{d i_2}{d t}
    \end{aligned}
\end{align*}

% TODO: Aggiungere diagrammi dei circuiti RL per la mutua induttanza
