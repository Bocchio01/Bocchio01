%!TEX TS-program = xelatex
%!TEX encoding = UTF-8 Unicode
% Awesome CV LaTeX Template for CV/Resume
%
% This template has been downloaded from:
% https://github.com/posquit0/Awesome-CV
%
% Author:
% Claud D. Park <posquit0.bj@gmail.com>
% http://www.posquit0.com
%
% Template license:
% CC BY-SA 4.0 (https://creativecommons.org/licenses/by-sa/4.0/)
%


%-------------------------------------------------------------------------------
% CONFIGURATIONS
%-------------------------------------------------------------------------------
% A4 paper size by default, use 'letterpaper' for US letter
\documentclass[11pt, a4paper]{awesome-cv}

\usepackage{float}
\usepackage{overpic}
\usepackage{tikz}
\usetikzlibrary{shapes.geometric}

% Configure page margins with geometry
\geometry{left=1.4cm, top=.8cm, right=1.4cm, bottom=1.8cm, footskip=.5cm}

% Color for highlights
% Awesome Colors: awesome-emerald, awesome-skyblue, awesome-red, awesome-pink, awesome-orange
%                 awesome-nephritis, awesome-concrete, awesome-darknight
\colorlet{awesome}{awesome-orange}
% Uncomment if you would like to specify your own color
% \definecolor{awesome}{HTML}{CA63A8}

% Colors for text
% Uncomment if you would like to specify your own color
% \definecolor{darktext}{HTML}{414141}
% \definecolor{text}{HTML}{333333}
% \definecolor{graytext}{HTML}{5D5D5D}
% \definecolor{lighttext}{HTML}{999999}
% \definecolor{sectiondivider}{HTML}{5D5D5D}

% Set false if you don't want to highlight section with awesome color
\setbool{acvSectionColorHighlight}{true}

% If you would like to change the social information separator from a pipe (|) to something else
\renewcommand{\acvHeaderSocialSep}{\quad\textbar\quad}

\hypersetup{
  pdftitle={Personal CV},
  pdfauthor={Tommaso Bocchietti},
  pdfsubject={Curriculum Vitae},
  pdfkeywords={}
}

%-------------------------------------------------------------------------------
%	PERSONAL INFORMATION
%	Comment any of the lines below if they are not required
%-------------------------------------------------------------------------------
% Available options: circle|rectangle,edge/noedge,left/right
% \photo[circle,noedge,left]{common/img/Profile}
\name{Tommaso}{Bocchietti}
\position{Junior Mechanical Engineer}
\address{Via Montagnola 13, San Fermo della Battaglia, Como (IT)}

\email{tommaso.bocchietti@gmail.com}
\mobile{(+39) 342-501-6560}
\github{Bocchio01}
\homepage{www.bocchio.dev}
\linkedin{tommaso-bocchietti}
% \dateofbirth{March 1st, 2001}
% \gitlab{gitlab-id}
% \stackoverflow{SO-id}{SO-name}
% \twitter{@twit}
% \skype{skype-id}
% \reddit{reddit-id}
% \medium{medium-id}
% \kaggle{kaggle-id}
% \googlescholar{googlescholar-id}{name-to-display}
%% \firstname and \lastname will be used
% \googlescholar{googlescholar-id}{}
% \extrainfo{extra information}

\quote{``Get things done!''}

\newcommand{\companyname}{\textbf{Whom It May Concern} }
\newcommand{\stars}[2][fill=darktext,draw=darktext]{
\begin{tikzpicture}[baseline=-0.34em,#1]
\foreach \X in {1,...,5}
{
\pgfmathsetmacro{\xfill}{min(1,max(1+#2-\X,0))}
\path (\X*1em,0)
node[star,draw, ultra thin,star point height=0.23em,minimum size=0.75em,inner sep=0pt,
path picture={\fill (path picture bounding box.south west)
rectangle  ([xshift=\xfill*0.722em]path picture bounding box.north west);}]{};
}
\end{tikzpicture}}

%-------------------------------------------------------------------------------
\begin{document}

% Print the header with above personal information
% Give optional argument to change alignment(C: center, L: left, R: right)
\makecvheader[C]

% Print the footer with 3 arguments(<left>, <center>, <right>)
% Leave any of these blank if they are not needed
\makecvfooter
{\today}
{Tommaso Bocchietti~~~·~~~Curriculum Vitae}
{\thepage}


%-------------------------------------------------------------------------------
%	CV/RESUME CONTENT
%	Each section is imported separately, open each file in turn to modify content
%-------------------------------------------------------------------------------
\descriptionstyle{
    Enthusiastic mechanical engineer with a strong problem-solving mindset and excellent technical skills in software development. \\
    My hands-on experience has enabled me to develop a diverse portfolio of projects, resulting in a distinctive ability to tackle complex challenges with a structured and analytical approach.
    I am currently seeking opportunities in the space industry to leverage my technical expertise, grow professionally in a dynamic environment, and contribute to cutting-edge exploration projects.
}

\cvsection{Experience}

\begin{cventries}

  % ------------------------------------

\cventry
{Robotics Researcher}
{Deutsches Zentrum für Luft- und Raumfahrt e.V. (DLR)}
{Oberpfaffenhofen, Munich (DE)}
{01.04.2025 - 19.12.2025}
{
    \begin{cvitems}
        \item {Developing a regrasping framework for the DLR's Hybrid Compliant Gripper (HCG).}
    \end{cvitems}
}
  % ------------------------------------

\cventry
{Mechatronic Engineer}
{Polimi Sailing Team}
{Politecnico di Milano, Milan (IT)}
{19.10.2023 - PRESENT}
{
    \begin{cvitems}
        \item {Coded from scratch high performance NMEA0183 libraries.}
        \item {Worked with STM32 microcontrollers to create interfaces for the sensors and the control algorithms.}
        \item {Selected as the overall winners of the 2024 edition of the \href{https://sumoth.org/}{``SuMoth''} competition among 11 participating teams.}
    \end{cvitems}
}
  % ------------------------------------

\cventry
{Mapper}
{Orienteering Como}
{Lombardy (IT)}
{02.02.2018 - PRESENT}
{
  \begin{cvitems}
    \item {Responsible for the society's cartography.}
    \item {Drawn several new Orienteering maps using the International Symbols Specification.}
    \item {Handled many homologation iter with the official Italian Federation for this sport.}
    \item {Developed a cloud-based architecture to facilitate the access and use of the map archive to all the members of the society.}
  \end{cvitems}
}
  % ------------------------------------

\cventry
{Private teacher}
{Self-Employed}
{Online \& Como (IT)}
{02.2018 - PRESENT}
{
    \begin{cvitems}
        \item {One-on-one tutoring and support in scientific subjects for students who are facing challenges.}
        \item {Students range in age from 14 to 19.}
    \end{cvitems}
}
  % ------------------------------------

\cventry
{Mechanical Engineer Intern}
{Politecnico di Milano in partnership with RIMAC Automobili}
{Politecnico di Milano, Milan (IT) \& RIMAC Automobili, Zagreb (HR)}
{09.03.2023 - 01.06.2023}
{
    \begin{cvitems}
        \item {Designed an innovative electric personal transportation vehicle concept aligned with RIMAC Automobili's vision for sustainable urban mobility.}
        \item {Developed 3D CAD models in OnShape and conducted structural FEA simulations to validate design integrity.}
        \item {Conducted ergonomic analysis and iterative design refinements to optimize user comfort, safety, and vehicle performance.}
        \item {Collaborated in an international team of 10 students from multiple European universities.}
        \item {Selected by company engineers as the best project among the 4 participating teams.}
    \end{cvitems}
}
  % ------------------------------------

\cventry
{Software developer}
{Confedilizia Como}
{Como (IT)}
{04.2020 - 02.2021}
{
    \begin{cvitems}
        \item {Responsible for the development of a custom software for the management of the real estate properties.}
        \item {Full stack development of the web platform for the secure distribution of the software to the clients.}
    \end{cvitems}
}
  % ------------------------------------

\cventry
{Web Programmer}
{Ennova Research}
{ComoNExT, Como (IT)}
{04.06.2018 - 13.06.2018}
{
    \begin{cvitems}
        \item {Period of internship for the "Alternanza Scuola Lavoro" national project.}
        \item {Developed a dynamically generated web page using Node.js and JavaScript.}
    \end{cvitems}
}

\end{cventries}
\cvsection{Education}

\begin{cventries}

  %---------------------------------------------------------
  \cventry
  {MSc in Mechanical Engineering (Mechatronics and Robotics)}
  {Politecnico di Milano}
  {Milan (IT)}
  {13.09.2023 - 07.2025 (expected graduation)}
  {
    \begin{cvitems}
      \item {Current GPA: 29.08/30}
      \item {Joined the "Polimi Sailing Team" in the "Mechatronics" department (A.Y. 2023/24)}
    \end{cvitems}
  }

  %---------------------------------------------------------
  \cventry
  {MSc in Mechanical Engineering (Erasmus+ exchange)}
  {University of Waterloo}
  {Waterloo (CA)}
  {01.01.2024 - 26.04.2024}
  {
    Relevant courses taken:
    \vspace{4mm}
    \begin{cvitems}
      \item {Advanced Finite Element Analysis: coded a 2D FEM solver for non-linear and plastic materials in MATLAB (A.Y. 2023/24)}
      \item {Computational Fluid Dynamics: coded a 2D CFD solver for incompressible fluid and a 1D solver for compressible fluid (A.Y. 2023/24)}
      \item {Materials for Nano and MEMS: complete a research project about Chip-Scale Atomic Clocks (A.Y. 2023/24)}
    \end{cvitems}
  }

  %---------------------------------------------------------
  \cventry
  {BSc in Mechanical Engineering}
  {Politecnico di Milano}
  {Milan (IT)}
  {14.09.2020 - 21.07.2023}
  {
    \begin{cvitems}
      \item {Selected for participating in the "Pro Hackin' Project 2023" (A.Y. 2022/23)}
      \item {Selected for competing at SWERC 2021 (A.Y. 2020/21)}
      \item {Third place in an internal coding competition using MATLAB (A.Y. 2020/21)}
    \end{cvitems}
  }

  %---------------------------------------------------------
  \cventry
  {High School Diploma, Scientific}
  {Scientific High School "Paolo Giovio"}
  {Como (IT)}
  {14.09.2015 - 06.2020}
  {
    \begin{cvitems}
      \item {Italian Physics Olympiad: admitted to regional selection (02.2019)}
      \item {Italian Informatics Olympiad: admitted to regional selection (04.2019, 04.2018)}
      \item {Italian Mathematics Olympiad: admitted to local district selection (02.2017)}
    \end{cvitems}
  }

\end{cventries}
\vspace{-\acvSectionTopSkip}

\begin{minipage}[t]{0.60\textwidth}

    \cvsection{Technical Skills}\\

    \begin{cvskills}

        \cvskill
        {Modeling \& Simulation}
        {MATLAB/Simulink, ROS, Model-Based Design}

        \cvskill
        {Embedded Systems}
        {STM32, ESP32, Arduino, Raspberry Pi}

        \cvskill
        {Programming}
        {C/C++, Python, Java | Web stack (JS, PHP, MySQL)}

        \cvskill
        {CAD \& Design}
        {CATIA V5, SolidWorks, Inventor}

        \cvskill
        {Tools}
        {Git, CI/CD, LaTeX, Linux, Windows}

    \end{cvskills}

\end{minipage}
%
\hfill
%
\begin{minipage}[t]{0.35\textwidth}

    \cvsection{Languages}\\

    \begin{cvskills}

        \cvskill
        {Italian}
        {Native}

        \cvskill
        {English}
        {Proficient}

        % \cvskill
        % {French}
        % {Beginner}

    \end{cvskills}

\end{minipage}

% Needed to correctly space vertically the successive section
%-------------------------------------------------------------------------------
%	SECTION TITLE
%-------------------------------------------------------------------------------
\cvsection{Extracurricular Activity}


%-------------------------------------------------------------------------------
%	CONTENT
%-------------------------------------------------------------------------------
\begin{cventries}

  %---------------------------------------------------------
  \cventry
  {IT Technician} % Affiliation/role
  {Italian Orienteering Committee} % Organization/group
  {Italy} % Location
  {Nov. 2018 - PRESENT} % Date(s)
  {
    \begin{cvitems} % Description(s) of experience/contributions/knowledge
      \item {Organizational IT aid at major Italian Orienteering Events.}
      \item {5 Days of Italy (Jul. 2022 - PRESENT)}
      \item {International MeetingOfVenice (Nov. 2018 - PRESENT)}
    \end{cvitems}
  }

  %---------------------------------------------------------
  \cventry
  {Events Organizer} % Affiliation/role
  {Orienteering Como} % Organization/group
  {Lombardy, Italy} % Location
  {Gen. 2017 - PRESENT} % Date(s)
  {
    \begin{cvitems} % Description(s) of experience/contributions/knowledge
      \item {Educational and promotional outings in the role of instructor (mainly for schools or local association).}
      \item {Organizer playing key roles (controller or course-setter) in smaller events such as promotional or regional competitions.}
    \end{cvitems}
  }

  %---------------------------------------------------------
  \cventry
  {Council Members} % Affiliation/role
  {Orienteering Como} % Organization/group
  {Como, Lombardy, Italy} % Location
  {Sept. 2021 - PRESENT} % Date(s)
  {
    \begin{cvitems} % Description(s) of experience/contributions/knowledge
      \item {Member since Feb. 2016}
      \item {Council Member since Sept. 2021}
      \item {An healthy way to keep myself doing sport.}
    \end{cvitems}
  }

  %---------------------------------------------------------
\end{cventries}

\cvsection{Honors \& Awards}

\begin{cvhonors}

  %---------------------------------------------------------
  \cvhonor
  {Merit Exemption}
  {Scholarship aimed at the group of top students based on GPA}
  {Milan (IT)}
  {2024}

  %---------------------------------------------------------
  \cvhonor
  {Merit Exemption}
  {Scholarship aimed at the group of top students based on GPA}
  {Milan (IT)}
  {2023}

  %---------------------------------------------------------
  \cvhonor
  {Merit Exemption}
  {Scholarship aimed at the group of top students based on GPA}
  {Milan (IT)}
  {2022}

  %---------------------------------------------------------
  \cvhonor
  {Merit Exemption}
  {Scholarship aimed at the group of top students based on GPA}
  {Milan (IT)}
  {2021}

  %---------------------------------------------------------
  \cvhonor
  {Best Freshman Award}
  {Scholarship aimed at the group of top freshmen students based on GPA}
  {Milan (IT)}
  {2021}

\end{cvhonors}
%-------------------------------------------------------------------------------
%	SECTION TITLE
%-------------------------------------------------------------------------------
\cvsection{Presentation}


%-------------------------------------------------------------------------------
%	CONTENT
%-------------------------------------------------------------------------------
\begin{cventries}

  %---------------------------------------------------------
  \cventry
  {Presenter for <CONTRATTI A CANONE CONCORDATO>} % Role
  {Online Webinar by F.I.M.A.A COMO} % Event
  {Online} % Location
  {16 Feb. 2021} % Date(s)
  {
    \begin{cvitems} % Description(s)
      \item {Explained the importance of a computer based system to efficiently generate the new type of real estate contract.}
      \item {Demonstrated the app developed for Confedilizia Como as a potential solution for compliance with new laws.}
    \end{cvitems}
  }

  %---------------------------------------------------------
\end{cventries}

\cvsection{Certifications}

\begin{cvskills}

  %---------------------------------------------------------
  \cvskill
  {Arduino 90/100}
  {Official Arduino certification, obtained on 17.04.2024}

  %---------------------------------------------------------
  \cvskill
  {TOEFL 90/120}
  {Test of English as a Foreign Language, obtained on 23.08.2023}

  %---------------------------------------------------------
  \cvskill
  {TOEIC 975/990}
  {Test of English for International Communication, obtained on 12.07.2023}

\end{cvskills}

%-------------------------------------------------------------------------------
%	SECTION TITLE
%-------------------------------------------------------------------------------
\cvsection{Hobby \& Personal interests}


%-------------------------------------------------------------------------------
%	CONTENT
%-------------------------------------------------------------------------------
\descriptionstyle{
  I love to go for adventures that push me both physically and mentally.\\
  For the past 8 years, I've been practice orienteering, a sport demanding map reading, physical effort and fast decision-making.
  Beyond that, I've done (and I'm planning for more) a couple of bikepacking journeys, conquering routes Como-London (1200km+) and Como-Barcelona (1100km+).
  These self-supported adventures showcase my tenacity, adaptability, and problem-solving skills in facing new situations and challenges.
}

\cvsection{Projects}

\descriptionstyle{
    Selection (not exhaustive) of some projects I've worked on that relate to my academic, professional, and personal interests.\\
    I consider them as way to experiment, learn, and get a hands-on approach to engineering.
}

\begin{cventries}

    \cventry
    {Most of the projects were done individually, at the explicit request of the professor.}
    {Academic Projects}
    {Politecnico di Milano, Milan (IT) \& University of Waterloo, Waterloo (CA)}
    {14.09.2020 - PRESENT}
    {
        \begin{cvitems}
            \item {\textbf{Modeling and Control of a MagLev system}: analysis of the dynamics of a magnetic levitation system, parameter identification and control/filters design (PID, LQR and MPC, coupled with KF and EKF). Simulation in MATLAB/Simulink and hardware deployment on \textit{RTDAC/PCI I/O} board from INTECO.}
            \item {\textbf{Study on Nonreciprocal Behavior in Time-Space Modulated Beams}: analysis of diode-like behavior in time-space modulated beams by means of piezoelectric shunts. Structure simulations in Comsol Multiphysics, experimental data analysis with MATLAB.}
            \item {\textbf{Structural Health Monitoring (SHM) as a multivariate outlier detection problem}: analysis of a tie-rods element subjected to both damage and environmental variability, by means of statistical indices as Mahalanobis Squared Distance (MSD) and Principal Component Analysis (PCA).}
            \item {\textbf{Topology Optimization of Hub Carrier}: mass minimization of a hub carrier structure, with constraints on the compliance and the manufacturability. Analysis with Altair HyperWorks suite.}
            \item {\textbf{Development of a 2D CFD solver in C/C++ for the solution of the Navier-Stokes equations for incompressible flows}: implementation of the SCGS and SIMPLE algorithms, with validation on the lid-driven cavity flow.}
        \end{cvitems}
        \vspace{9pt}
    }

    \cventry
    {Pro Hackin' Project 2023, in partnership with RIMAC Automobili}
    {Personal Transportation Vehicle (Sidewalk vehicle)}
    {Politecnico di Milano, Milan (IT) \& RIMAC Automobili, Zagreb (HR)}
    {09.03.2023 - 01.06.2023}
    {
        \textbf{Idealize and Design a Personal Transportation Vehicle (Sidewalk vehicle).}
        \newline
        \vspace{2mm}
        Team-based project involving students from 4 top European universities, with corporate feedback sessions after each hackathon.
        Selected as winning team by RIMAC Automobili engineers.
        \newline
        \vspace{4mm}
        \begin{cvitems}
            \item {Hackathon 1: visions, user personas and functions \& requirements}
            \item {Hackathon 2: functional decomposition, morphological matrix and concepts development}
            \item {Hackathon 3: CAD design, FEM simulations, FMEA and cost analysis.}
        \end{cvitems}
        \vspace{9pt}
    }

    \cventry
    {Personal project to celebrate BSc}
    {Model Rocket with On-Board Flight Computer}
    {Personal Workshop, Como (IT)}
    {12.07.2023 - 21.07.2023}
    {
        \textbf{Design, optimization and realization of a $63cm$ model rocket.}
        \newline
        \vspace{2mm}
        Final design was achieved after a couple of iterations between CAD model and CFD simulations.
        Built mostly from cheap materials (cardboard \& wood) and 3D printed parts (PLA based).
        Essential characteristics:
        \newline
        \vspace{4mm}
        \begin{cvitems}
            \item {Launched and recovered without any damage, maximum elevation $+538m$}
            \item {Flight time $\approx 60s$, maximum speed reached $+120m/s$, maximum acceleration $+10g$.}
            \item {Simulations driven design ($C_D \approx 0.806$)}
            \item {On board flight computer in $< 20cm^3$, with barometric, temperature and acceleration sensors capable of logging data.}
        \end{cvitems}
    }

\end{cventries}


% Signature and data treatment
\vspace*{\fill}

\begin{flushright}
  \descriptionstyle{\textit{Tommaso Bocchietti}}

  \includegraphics[width=0.25\textwidth]{common/img/Signature.png} \vspace{-4mm} \\
  \rule{0.27\textwidth}{0.5pt}
\end{flushright}

\begin{center}
  \footerstyle{I hereby give consent for my data included in this document to be processed by \companyname for recruiting purposes.}
\end{center}

%-------------------------------------------------------------------------------
\end{document}
