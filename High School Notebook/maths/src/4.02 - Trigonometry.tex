% LTeX: language=it

\section{Trigonometria}

La trigonometria è lo studio delle relazioni tra i lati e gli angoli di un triangolo.

% TODO: disegno triangolo rettangolo

\begin{figure}[H]
    \begin{minipage}[b]{0.45\textwidth}
        \centering
        \begin{tikzpicture}
            % Triangle
            \coordinate (A) at (0,0);
            \coordinate (B) at (5,0);
            \coordinate (C) at (5,2.5);

            % Angle names
            % \node at (A) [below, left] {$\beta$};
            % \node at (B) [below, right] {$\gamma$};
            % \node at (C) [above, right] {$\alpha$};

            \pic [draw, -, "$\gamma$"] {angle = C--B--A};
            \pic [draw, -, "$\alpha$"] {angle = A--C--B};
            \pic [draw, -, "$\beta$", angle radius=1cm] {angle = B--A--C};

            % Edge labels
            \draw (A) -- (B) node[midway, below] {$a$};
            \draw (B) -- (C) node[midway, right] {$b$};
            \draw (C) -- (A) node[midway, above] {$c$};

        \end{tikzpicture}
        \caption{Triangolo rettangolo}
        \label{fig:triangle}
    \end{minipage}
    \hfill
    \begin{minipage}[b]{0.45\textwidth}
        \vspace{0pt}
        \begin{align*}
            a = \begin{cases}
                    c \sin(\alpha) \\
                    c \cos(\beta)  \\
                    b \tan(\alpha) \\
                    b \cot(\alpha) \\
                \end{cases}
        \end{align*}
        Preso per esempio il lato $a$, si possono scrivere le suddette relazioni.
    \end{minipage}
\end{figure}

Risolvere un triangolo significa trovare il valore di ogni suo lato e angolo.

\subsection{Teoremi sui triangoli rettangoli}

\begin{itemize}
    \item Area di un triangolo: $A = \frac{1}{2}ab\sin(\alpha)$ (due lati per l'angolo compreso) % TODO: disegno triangolo rettangolo
    \item Misura di una corda: $AB = 2r\sin(\alpha)$ % TODO: disegno triangolo rettangolo inscritto circonferenza
    \item Raggio della circonferenza inscritta: $r = \frac{a}{2\sin(\alpha)}$
\end{itemize}

\subsection{Teoremi sui triangoli qualunque}

\begin{itemize}
    \item Teorema dei seni: $\frac{a}{\sin(\alpha)} = \frac{b}{\sin(\beta)} = \frac{c}{\sin(\gamma)}$ % TODO: disegno triangolo rettangolo
    \item Teorema del coseno: $a^2 = b^2 + c^2 - 2bc\cos(\alpha)$, dimostrabile essendo $a^2 = HH'^2 + (AB-HB)^2 = (b\sin(\alpha))^2 + c^2 + (b\cos(\alpha))^2 - 2bc\cos(\alpha)$
\end{itemize}