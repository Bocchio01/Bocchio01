\cvsection{Projects}

\descriptionstyle{
    Selection (not exhaustive) of some projects I've worked on that relate to my academic, professional, and personal interests.\\
    I consider them as way to experiment, learn, and get a hands-on approach to engineering.
}

\begin{cventries}

    \cventry
    {Most of the projects were done individually, at the explicit request of the professor.}
    {Academic Projects}
    {Politecnico di Milano, Milan (IT) \& University of Waterloo, Waterloo (CA)}
    {14.09.2020 - PRESENT}
    {
        \begin{cvitems}
            \item {\textbf{Structural Health Monitoring (SHM) as a multivariate outlier detection problem}: analysis of a tie-rods element subjected to both damage and environmental variability, by means of statistical indices as Mahalanobis Squared Distance (MSD) and Principal Component Analysis (PCA).}
            \item {\textbf{Development of a 2D CFD solver in C/C++ for the solution of the Navier-Stokes equations for incompressible flows}: implementation of the SCGS and SIMPLE algorithms, with validation on the lid-driven cavity flow.}
            \item {\textbf{Chip Scale Atomic Clocks (CSAC)}: analysis of the physics behind their operation and current state of the art, with a focus on MEMS/NEMS technology.}
        \end{cvitems}
    }

    \cventry
    {Pro Hackin' Project 2023, in partnership with RIMAC Automobili}
    {Personal Transportation Vehicle (Sidewalk vehicle)}
    {Politecnico di Milano, Milan (IT) \& RIMAC Automobili, Zagreb (HR)}
    {09.03.2023 - 01.06.2023}
    {
        \textbf{Idealize and Design a Personal Transportation Vehicle (Sidewalk vehicle).}
        \newline
        \vspace{2mm}
        Team-based project involving students from 4 top European universities, with corporate feedback sessions after each hackathon.
        Selected as winning team by RIMAC Automobili engineers.
        \newline
        \vspace{4mm}
        \begin{cvitems}
            \item {Hackathon 1: visions, user personas and functions \& requirements}
            \item {Hackathon 2: functional decomposition, morphological matrix and concepts development}
            \item {Hackathon 3: CAD design, FEM simulations, FMEA and cost analysis.}
        \end{cvitems}
    }

    \cventry
    {Personal project to celebrate BSc}
    {Model Rocket with On-Board Flight Computer}
    {Personal Workshop, Como (IT)}
    {12.07.2023 - 21.07.2023}
    {
        \textbf{Design, optimization and realization of a $63cm$ model rocket.}
        \newline
        \vspace{2mm}
        Final design was achieved after a couple of iterations between CAD model and CFD simulations.
        Built mostly from cheap materials (cardboard \& wood) and 3D printed parts (PLA based).
        Essential characteristics:
        \newline
        \vspace{4mm}
        \begin{cvitems}
            \item {Launched and recovered without any damage, maximum elevation $+538m$}
            \item {Flight time $\approx 60s$, maximum speed reached $+120m/s$, maximum acceleration $+10g$.}
            \item {Simulations driven design ($C_D \approx 0.806$)}
            \item {On board flight computer in $< 20cm^3$, with barometric, temperature and acceleration sensors capable of logging data.}
        \end{cvitems}
    }

\end{cventries}
