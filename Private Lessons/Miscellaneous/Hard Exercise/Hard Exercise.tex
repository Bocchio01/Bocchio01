\documentclass[a4paper, 12pt]{article}

\usepackage[italian]{babel}

\usepackage{tikz}
\usepackage{pgfplots}
\pgfplotsset{compat=1.18}
\usepackage[hidelinks]{hyperref}
\usepackage{amsmath}
\usepackage{float}


\hypersetup{
pdftitle={Problema difficile},
pdfsubject={Auguri di compleanno per Alessia},
pdfauthor={Tommaso Bocchietti}
}

\title{Problema difficile}
\author{Tommaso Bocchietti}
\date{08/05/2023}

\graphicspath{{./img/}{./pdf/}}


\begin{document}

\begin{figure}
    \centering
    \includegraphics[width=.9\textwidth]{logo_polimi}
\end{figure}

\maketitle

\clearpage
% \listoffigures
% \listoftables

\tableofcontents
\pagebreak

% LTeX: language=it

\section{Testo del Problema}

Il problema proposto è il seguente:

Considerando un normale autobus di linea da circa 50 posti, quanti palline da ping pong possono starci all'interno?

Ogni dato non specificato può essere considerato a piacere.

\pagebreak

\section{Soluzione}
Si procede ora alla risoluzione dei tre problemi proposti.

% LTeX: language=it

\subsection{Problema 1}
Possiamo inizialmente visualizzare la curva di nostro interesse in un piano cartesiano tridimensionale:
\begin{center}
    \begin{tikzpicture}
        \begin{axis}[
                view={30}{10},
                axis lines=center,
            ]
            \addplot3+[domain=0:sqrt(7),samples=60,samples y=0]
            ({cos(deg(x))},
            {sin(deg(x))},
            {sqrt(6)*x});
        \end{axis}
    \end{tikzpicture}

    \begin{tikzpicture}
        \begin{axis}[
                view={0}{90},
                axis lines=center,
                xlabel=$x$,
                ylabel=$y$,
                zlabel=$z$,
                axis equal
            ]
            \addplot3+[domain=0:sqrt(7),samples=60,samples y=0]
            ({cos(deg(x))},
            {sin(deg(x))},
            {sqrt(6)*x});
        \end{axis}
    \end{tikzpicture}
\end{center}

\noindent Si procede ora al calcolo della lunghezza della curva come:
\begin{equation}
    \int_{0}^{\sqrt{7}} \sqrt{\left(\frac{dx}{dt}\right)^2 + \left(\frac{dy}{dt}\right)^2 + \left(\frac{dz}{dt}\right)^2} dt
\end{equation}
Si procede ora al calcolo delle derivate:
\begin{equation}
    \begin{split}
        \frac{dx}{dt} = -\sin(t) \\
        \frac{dy}{dt} = \cos(t)  \\
        \frac{dz}{dt} = \sqrt{6}
    \end{split}
\end{equation}
Si procede ora al calcolo della lunghezza della curva:
\begin{equation}
    \begin{split}
        \int_{0}^{\sqrt{7}} \sqrt{(-\sin(t))^2 + (\cos(t))^2 + (\sqrt{6})^2} dt = \\
        = \int_{0}^{\sqrt{7}} \sqrt{1 + 6} dt = \sqrt{7} \int_{0}^{\sqrt{7}} dt = \sqrt{7} \sqrt{7} = 7
    \end{split}
\end{equation}
% LTeX: language=it

\subsection{Problema 2}
Come prima, è possibile visualizzare i due vettori e il vettore prodotto in un piano cartesiano tridimensionale.
Si sottolinea come la figura seguente non rappresenti la soluzione del problema, ma sia solo un aiuto visivo per la comprensione del problema.
\begin{center}
    \begin{tikzpicture}[x=1cm, y=1cm, z=-0.6cm]
        % Axes
        \draw [->] (-3,0,0) -- (4,0,0) node [right] {$x$};
        \draw [->] (0,0,0) -- (0,4,0) node [left] {$y$};
        \draw [->] (0,0,0) -- (0,0,4) node [left] {$z$};
        % Vectors
        \draw [->, thick] (0,0,0) -- (2,2,0);
        \draw [->, thick] (0,0,0) -- (2,0,1);
        \draw [->, thick, red] (0,0,0) -- (-2,2,4);
        % Ticks
        \foreach \i in {1, 2, 3}
            {
                \draw (-0.1,\i,0) -- ++ (0.2,0,0);
                \draw (\i,-0.1,0) -- ++ (0,0.2,0);
                \draw (-0.1,0,\i) -- ++ (0.2,0,0);
            }
        \draw (-1,-0.1,0) -- ++ (0,0.2,0);
        \draw (-2,-0.1,0) -- ++ (0,0.2,0);
        % Dashed lines
        \draw [loosely dashed]
        (0,2,0) -- (2,2,0) -- (2,0,0)
        (0,0,1) -- (2,0,1) -- (2,0,0)
        (0,0,4) -- (-2,0,4) -- (-2,2,4)
        (-2,0,0) -- (-2,0,4)
        ;
        % Labels
        \node [right] at (2,2,0) {$\begin{bmatrix}
                    2 \\2\\0
                \end{bmatrix}$};
        \node [below] at (2,0,1) {$\begin{bmatrix}
                    2 \\0\\1
                \end{bmatrix}$};
        \node [above] at (-2,2,4) {$\begin{bmatrix}
                    -2 \\2\\4
                \end{bmatrix}$};

    \end{tikzpicture}
\end{center}

\noindent Si procede ora al calcolo del prodotto vettoriale tra $\vec{a}$ e $\vec{b}$:
\begin{equation}
    \vec{a} \times \vec{b} = \begin{vmatrix}
        \vec{i} & \vec{j}     & \vec{k}               \\
        1       & \frac{1}{3} & 3+6\sqrt{\frac{2}{5}} \\
        3       & 1           & 9
    \end{vmatrix} = \vec{i} \begin{vmatrix}
        \frac{1}{3} & 3+6\sqrt{\frac{2}{5}} \\
        1           & 9
    \end{vmatrix} - \vec{j} \begin{vmatrix}
        1 & 3+6\sqrt{\frac{2}{5}} \\
        3 & 9
    \end{vmatrix} + \vec{k} \begin{vmatrix}
        1 & \frac{1}{3} \\
        3 & 1
    \end{vmatrix}
\end{equation}
Si procede ora al calcolo dei determinanti:
\begin{equation}
    \begin{split}
        \begin{vmatrix}
            \frac{1}{3} & 3+6\sqrt{\frac{2}{5}} \\
            1           & 9
        \end{vmatrix} = \frac{1}{3} * 9 - (3+6\sqrt{\frac{2}{5}}) = -6\sqrt{\frac{2}{5}} \\
        \begin{vmatrix}
            1 & 3+6\sqrt{\frac{2}{5}} \\
            3 & 9
        \end{vmatrix} = 1 * 9 - 3(3+6\sqrt{\frac{2}{5}}) = -18\sqrt{\frac{2}{5}} \\
        \begin{vmatrix}
            1 & \frac{1}{3} \\
            3 & 1
        \end{vmatrix} = 1 * 1 - \frac{1}{3} * 3 = 1 - 1 = 0
    \end{split}
\end{equation}
Si procede ora al calcolo del prodotto vettoriale:
\begin{equation}
    \vec{a} \times \vec{b} = \vec{i} (-6\sqrt{\frac{2}{5}}) - \vec{j} (-18\sqrt{\frac{2}{5}}) + \vec{k} * 0 = (-6\sqrt{\frac{2}{5}})\vec{i} + (18\sqrt{\frac{2}{5}})\vec{j}
\end{equation}
Si procede ora al calcolo del modulo del prodotto vettoriale:
\begin{equation}
    |\vec{a} \times \vec{b}| = \sqrt{(-6\sqrt{\frac{2}{5}})^2 + (18\sqrt{\frac{2}{5}})^2} = \sqrt{36\frac{2}{5} + 324\frac{2}{5}} = \sqrt{\frac{720}{5}} = \sqrt{144} = 12
\end{equation}
% LTeX: language=it

\subsection{Problema 3}
Osserviamo graficamente il problema dato e notiamo si tratta di un limite elementare non essendoci punti di discontinuità:
\begin{center}
    \begin{tikzpicture}
        \begin{axis}[
                axis lines=middle,
                xmin=0,xmax=10,
                ymin=-0.1,ymax=0.2,
                legend style={anchor=north west},
            ]
            \addplot[blue, samples=200, domain=0.01:10] {ln(x^(4*sqrt(3)/73))};
            \addlegendentry{$\ln(x^{\frac{4\sqrt{3}}{73}})$}
            \addplot[red, samples=200, domain=0.01:10] {4*sqrt(3)/(73*x)};
            \addlegendentry{$\frac{d}{dx}\ln(x^{\frac{4\sqrt{3}}{73}}) = \frac{4\sqrt{3}}{73x}$}
            \addplot[only marks, mark=*] coordinates {(8.66, 4/365)};
        \end{axis}
    \end{tikzpicture}
\end{center}

\noindent Si procede ora al calcolo del limite:
\begin{equation}
    \lim_{x \rightarrow 5\sqrt{3}} \frac{d}{dx} \ln(x^{\frac{4\sqrt{3}}{73}})
\end{equation}
Si procede ora al calcolo della derivata:
\begin{equation}
    \frac{d}{dx} \ln(x^{\frac{4\sqrt{3}}{73}}) = \frac{4\sqrt{3}}{73} \frac{d}{dx} \ln(x) = \frac{4\sqrt{3}}{73} \frac{1}{x}
\end{equation}
Si procede ora al calcolo del limite:
\begin{equation}
    \lim_{x \rightarrow 5\sqrt{3}} \frac{4\sqrt{3}}{73} \frac{1}{x} = \frac{4\sqrt{3}}{73} \frac{1}{5\sqrt{3}} = \frac{4}{365}
\end{equation}

% LTeX: language=it

\section{Risultato}
Il risultato dei tre problemi proposti è:
\begin{itemize}
    \item $\int_{\gamma} f(t) dt = 7$
    \item $|\vec{a} \times \vec{b}| = 12$
    \item $\lim_{x \rightarrow 5\sqrt{3}} \frac{d}{dx} \ln(x^{\frac{4\sqrt{3}}{73}}) = \frac{4}{365}$
\end{itemize}
Supponendo poi una unità di misura del risultato pari a $[anni]$, si ha:
\begin{equation}
    7 + 12 + \frac{4}{365} = 19 + \frac{4}{365} \space [anni]
\end{equation}
La soluzione del problema rispecchia così l'età di una delle due risolutrici\dots

\begin{center}
    \huge Auguri di Buon Compleanno Alessia!
\end{center}

\end{document}
